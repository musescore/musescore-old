%====================================================================
%
%     MuseScore Manual
%     this is the MuseScore manual
%
% (C) 2005 Copyright: Werner Schweer
%====================================================================

%----------------------------------------------------------
%     Setup
%----------------------------------------------------------

%\showframe
%\showsetups
\usemodule[chart]
\setupcolors[state=start]
\setupbodyfont[Palatino]
\language[en]
\enableregime[il1]

\startuseMPgraphic{FunnyFrame}
      picture p; numeric o; path a, b; pair c;
      p := textext.rt(\MPstring{FunnyFrame});
      o := BodyFontSize;
      a := unitsquare xyscaled(OverlayWidth,OverlayHeight);
      p := p shifted (20, OverlayHeight-ypart center p);
      drawoptions (withpen pencircle scaled 1pt withcolor .625red);
      b := a superellipsed .95;
      fill b withcolor .85white; draw b;
      b := (boundingbox p) superellipsed .95;
      fill b withcolor .85white; draw b;
      draw p withcolor black;
      setbounds currentpicture to a;
    \stopuseMPgraphic

\defineoverlay[FunnyFrame][\useMPgraphic{FunnyFrame}]
\defineframedtext[FunnyText][frame=off,background=FunnyFrame]
\def\StartFrame{\startFunnyText}
\def\StopFrame{\stopFunnyText}
\def\FrameTitle#1%
 {\setMPtext{FunnyFrame}{\hbox spread 1em{\hss\strut#1\hss}}}
\setMPtext{FunnyFrame}{}


\define\M{MuseScore}
\define[1]\Index{{\it #1}\marginpar{#1}\index{#1}}

\defineindenting[Cmdi][text=,separator=,width=fit,distance=1em]

\define[1]\Cmd{
  \Cmdi
  \framed[
     background=color,
     width=fit,
     align=right,
     backgroundcolor=lightgray,
     framecolor=blue]
     {\tt\space #1}}

\define[2]\CCmd{
  \Cmdi
  \framed[
     background=color,
     width=broad,
     align=right,
     backgroundcolor=lightgray,
     framecolor=blue]
     {\vbox{\hbox{\tt\space #1}\hbox{\tt\space #2}}}}


%\define\startdescription{\startpacked}
\define\startdescription{}
%\define\stopdescription{\stoppacked}
\define\stopdescription{}

\define[1]\Fig{\hbox{\externalfigure[pics/#1]}}

\define[2]\Screen{
      \placefigure[here][fig:#1]{#2}{\externalfigure[pics/#1][scale=1600]}
      }
\define[1]\Noten{
      \blank[line]
      {\externalfigure[pics/#1][scale=700]}
      \blank[line]
      }

\definedescription[Option][
      location=left,
      headstyle=bold,
      width=4em,
      before={\startnarrower[left]\setupblank[0pt]},
      after={\stopnarrower\setupblank}]

\definedescription[Sc][
      location=left,
      headstyle=bold,
      width=4em,
      before={\startnarrower[left]\setupblank[0pt]},
      after={\stopnarrower\setupblank}]

\definedescription[Filetype][
      location=left,
      headstyle=bold,
      width=4em,
      before={\startnarrower[left]\setupblank[0pt]},
      after={\stopnarrower\setupblank}]

\definedescription[FileList][
      location=left,
      headstyle=bold,
      width=7em,
      before={\startnarrower[left]\setupblank[0pt]},
      after={\stopnarrower\setupblank}]

\definedescription[Input][
      location=left,
      headstyle=bold,
      width=9em,
      before={\startnarrower[left]\setupblank[0pt]},
      after={\stopnarrower\setupblank}]

\definedescription[InputN][
      location=left,
      headstyle=bold,
      width=6em,
      before={\startnarrower[left]\setupblank[0pt]},
      after={\stopnarrower\setupblank}]

\define[2]\Figure{
      \placefigure
         [#1][fig:#2]{}
         {\externalfigure[pics/#2]}
         }

\define[1]\Menu{
      \placefigure[right][fig:#1]{}
         {\externalfigure[pics/#1][scale=2000]}
         }

\definedescription[Opt][location=hanging,headstyle=bold,width=broad]

%----------------------------------------------------------
%     Body
%----------------------------------------------------------

\starttext
   \language[en]
   \mainlanguage[en]
   \startstandardmakeup[doublesided=no]
      \definebodyfont[10pt,11pt,12pt][rm][tfe=Regular at 48pt]
      \tfe\setupinterlinespace
      \hfill \color[red]{\M}\par
      \hfill \color[blue]{Manual}\par
      \vfill
      \hfill \color[blue]{EN}\par
      \definebodyfont[10pt,11pt,12pt][rm][tfb=Regular at 24pt]
      \tfb\setupinterlinespace
      \hfill Version 0.1\par
      \stopstandardmakeup

   \startstandardmakeup[page=no]
      \vfill
      \M\ is hosted on SourceForge:\par
      \type{http://mscore.sourceforge.net/}
      \blank[line]
      \M\ uses the Lilypond Typesetter fonts.\par
      This document was created using \pdfTeX\ and the macro package
      \ConTeXt\.
      \blank[line]
      \copyright 2005 Werner Schweer
      \stopstandardmakeup

   \completecontent

\chapter{Introduction}
   \section{What is \M?}
      \M\ is a WYSIWYG (What You See Is What You Get)
      program to create printed score.

      \M\ is "open source" and published under the GNU General
      Public License (GPL) (see Appendix).

      Some highlights:

      \startitemize[packed]
            \item WYSIWYG Design, notes are entered on a "virtual notesheet"
            \item \M\ uses TrueType fonts for printing and screen display.
                  This allows for high quality renderings on all magnification steps.
            \item Notes can be entered fast and simply by only using the keyboard.
            \item MusicXML import/export
            \item Midi import/export
            \item Midi input for note entry
            \item integrated Fluid software synthesizer
      \stopitemize

   \section{Short Guide}
      \subsection{Program Start}

      \Screen{main1}{Empty main window}

      After entering

      \Cmd{mscore}

      the last project will be displayed. Starting \M\ the first time an empty
      main window will be displayed.
      (Abb. \in[fig:main1]).

   The window has three areas:

   \startitemize[packed]
      \item menu bar
      \item tool bar
      \item empty note canvas
   \stopitemize

  \subsection{Create a new score}

      To create a new note sheet enter \type{Strg+N}.
      A dialog containing a list of templates to select from appears.
      (Abb. \in[fig:template]).
      \index{Templates}

      \Screen{template}{Template selection}

      After selecting \type{a4piano.msc.bz2} we leave the dialog by entering
      \type{OK}.

      On the main window canvas a worksheet with the selected template will be
      displayed, in our example two note lines connected with a brace.
      A clef is displayed and the time signature is set to $4/4$ on default.
      The note lines are populated with some measures filled with rests.

      \Screen{emptypiano}{Empty piano system}

\subsection{Note Entry}

      Now we want to enter some notes. For this we select the first rest by
      clicking at it. Then we enter "note entry mode" by entering
      \type{N}. The cursor changes to show the entry mode.
      An insert cursor also appears.

      To populate the empty measures with notes we simply type
      ''c d e'' on the keyboard.

      \Noten{noten1.pdf}

\subsection{Edit}
   Some actions:
   \startitemize[packed]
      \item you can move the note sheet on the canvas with pressed left mouse key
      \item pressing \type{Strg+M} appends a new empty measure to the score
      \item the magnify glass tool magnifies the canvas
   \stopitemize

\subsection{Print}

      Pressing the print button starts the KDE print dialog. \M\ creates
      postscript printer data which KDE sends to the printer or which
      can be redirected into a file. This also allows for direct creation
      of pdf files.

      \FrameTitle{Tip:}
      \StartFrame
      To create score examples for a book, create a partitur page with
      a small page format as A5 and rotate the page (landscape option).
      After exporting the score to PDF, the empty margins can be removed
      with the utility \type{pdfcrop}. The result can be easily integrated
      in a  \ConTeXt\ document. This is how the score examples in this manual
      are  created.
      \StopFrame

%----------------------------------------------------------
%
%     Referenz
%
%----------------------------------------------------------

\chapter{Reference}
   \section{Program Start}

      \M\ can be started from the commandline with
      \index{commandline}

            \Cmd{mscore <options> <file name>}

      \type{options} und \type{file name} are optional.

      \index{Comman line options}\index{options}\index{version number}
      \index{debug mode}

      There are the following command line options:
            \blank[medium]
            \Option{ -v} displays the \M\ version number\par
            \Option{ -d} start \M\ in debug mode\par
            \Option{ -s} disable the integrated software synthesizer\par
            \Option{ -m} disable midi input support\par
            \Option{ -h} help: show available options\par
            \blank[big]

      \blank[line]

      \M\ accepts the following file types:
            \blank[medium]
            \Filetype{*.msc}  \M\ partitur file\par
            \Filetype{*.mss}  \M\ style file\par
            \Filetype{*.xml}  MusicXML file\par
            \Filetype{*.mid}  midi file\par
            \blank[big]
      \index{file types}

      Starting \M\ without file name loads the last edited partitur.

      \M\ writes/reads two additional files in the backgrounde:
            \blank[medium]
            \FileList{\textasciitilde/.mscore} \M\ configuration data
            \FileList{\textasciitilde/.mscorePrj} contains the last edited scores
            \blank[big]

   \section{Menus}
      Many menu options are also available as buttons in a toolbox.

      Menus and keyboard shortcuts:

            \blank[medium]
            \Option{Alt+F} File\par
            \Option{Alt+E} Edit\par
            \Option{Alt+C} Create\par
            \Option{Alt+N} Notes\par
            \Option{Alt+L} Layout\par
            \Option{Alt+S} Style\par
            \Option{Alt+D} Display\par
            \Option{Alt+H} Help\par
            \blank[big]


      \subsection{File Menu}

      \Menu{menu1}

      \startpacked
      \Opt{New (Ctrl+N)} creates a new note sheet. You have next to create
            instruments and some empty measures.\par
      \Opt{Open} opens a saved score\par
      \Opt{Open Recent} shows a list of last edited scores. Click on a score
            to open it.\par
      \Opt{Save} save the current score to disc.\par
      \Opt{Save As} saves the current score to disc with a selectable different name.\par
      \Opt{Export Midi} export the current score as MIDI file.
            \index{Midi-Datei}\par

      \Opt{Export MusicXML} export the current score as
        MusicXML file. \index{MusicXML}

      \Opt{Import Midi} import a midi file and display as score.

      \Opt{Import MusicXML} import MusicXML file

      \Opt{Print (Ctrl+P)} print the current score
        \index{printing}

      \Opt{Quit} quits \M.\par
        \index{quit}

      \stoppacked


      \subsection{Edit Menu}

      \Menu{menu2}

      \startpacked
      \Opt{Undo}  undo the last edit. There are unlimited undos.

      \Opt{Redo}  ''Undo'' the last undo command.

      \Opt{Cut}

      \Opt{Copy}

      \Opt{Paste}

      \Opt{Instrument List}   shows the instrument list.

      \Opt{Page List}   shows the page list. This is a debug options useful for debugging.
            The page list is a \M\ internal data structure.

      \Opt{Preferences} shows the preference dialog.\par

      \stoppacked


      \subsection{Create Menu}

            \Menu{menu3}

      \startpacked
            \Opt{Instruments}
                  opens the instrument dialog. You can add/remove/move
                  instruments of your score in the instrument dialog.
                  You can also add addtional note lines to an instrument.
                  \par
            \Opt{Measure}
                  appends an empty measure to the score. The measure is filled
                  with a rest.\par
            \Opt{Clef}
                  shows the clefs palette\par
            \Opt{Key}
                  shows the keys palette.\par
            \Opt{Time}
                  shows the time signature palette.\par
            \Opt{Lines}
                  shows the lines palette(crescendo etc.)\par
            \Opt{Note Attributes}
                  shows the note attributes palette\par
            \Opt{Dynamics}
                  shows the dynamics palette\par
            \Opt{Text}
                  shows the text submenu\par
            \Opt{Symbols}
                  shows the symbol palette\par
            \stoppacked

      \subsection{Notes Menu}

            \Menu{menu4}

         \startpacked

            \Opt{Input}
                  switches to note entry mode
                  \index{note entry mode}\par
            \Opt{Add Pitch}
                  shows another submenu with note values to enter
                  \par
            \Opt{Add Intervall}
                  shows a list of intervalls to select from to
                  create chords\par
            \Opt{N-Tole}
                  shows a submenu to create irregular note values
                  (trioles pentoles etc.)\par
            \stoppacked

         \startcombination[3*1]
            {\externalfigure[pics/menu41][scale=1200]}{}
            {\externalfigure[pics/menu42][scale=1200]}{}
            {\externalfigure[pics/menu43][scale=1200]}{}
            \stopcombination

      \subsection{Layout Menu}

            \Menu{menu5}

      \startpacked
            \Opt{PageSettings}
                  shows the page settings dialog.\par
            \Opt{Reset Positions}
                  resets all marked objects to there standard positions.
                  This undos all manuell moving of objects.
                  This also flips note stem direction back to
                  default.\par
            \Opt{Set Normal Staff Distances}
                  resets modified note line distances back to there
                  standard values.\par
            \Opt{Reset Stretch}
                  resets stretched measures back to there normal width\par
            \Opt{System Break}
                  creates a line break after the marked measure\par
            \Opt{Page Break}
                  creates a page break after the marked measure\par
            \stoppacked


      \subsection{Style Menu}

            \Menu{menu6}

      \startpacked
            \Opt{Edit Style}
                  shows the style editor\par
            \Opt{Edit Text Style}
                  shows the text style editor\par
            \Opt{Load Style}
                  load a new style from file.\par
            \Opt{Save Style}
                  writes the current style to disc\par
            \stoppacked


      \subsection{Display Menu}

            \Menu{menu7}

      \startpacked
            \Opt{Pad}
                  show the input pad\par
            \Opt{Play Panel}
                  show the play panel\par
             \Opt{Transport Toolbar}
                  show the transport toolbar\par
            \Opt{Show Invisible}
                  If you switch this option on, invisible note elements are
                  displayed in gray. Dies enables editing.\par
            \stoppacked

      \subsection{Help Menu}

            \Menu{menu8}

      \startpacked
            \Opt{Browser}
                  starts the configured online help browser\par
            \Opt{About}
                  show the about panel\par
             \Opt{About Qt}
                  show infos about the Trolltech Qt GUI-Tookit.\par
            \Opt{Whats This}
                  after clicking the ''whats this'' cursor appears.
                  Clicking on any GUI-Element shows a help text for this
                  element (if any help text is assigned to this element).
                  \par
            \stoppacked

%----------------------------------------------------------
%     Note Entry
%----------------------------------------------------------

   \section{Note Entry}

      \M\ is always in one of two modes:

      \startitemize[packed]
            \item {\bf edit mode}, (normal) shows a normal arrow cursor
            \item {\bf input mode}, shows an up arrow cursor
      \stopitemize

            \blank[medium]
            \Input{N} begin note input mode\par
            \Input{Esc} end note input mode\par
            \blank[big]

      \FrameTitle{Hint:}
      \StartFrame
      In input mode all pad or toolbar settings are for the next note to enter.
      In normal mode changing pad or toolbar settings directly modify the
      selected note.
      \StopFrame

      Entry mode is activated by clicking a note in the pad or by
      typing \type{N}.

      \subsection{Keyboard entry}

      Note can be entered by typing
            \blank[medium]
            \Input{a b c d e f g}\par
            \blank[big]

      Notes are entered at the current position. The current position
      ca be set by clicking at a note or rest is is displayed by a
      blinking cursor. The cursor position is always in front of the
      current position. The len of the current note and other properties
      can be set in the pad or the toolbar. The pad simulates the keys
      of the numeric keypad of a normal keyboard.

      When a note is selected, the next command can create chords. Every
      input switches into the note entry mode:

            \blank[medium]
            \InputN{Shift+A} add note A to akkord\par
            \InputN{Shift+B} add note B to akkord\par
            \InputN{Shift+C} add note C to akkord\par
            \InputN{Shift+D} add note D to akkord\par
            \InputN{Shift+E} add note E to akkord\par
            \InputN{Shift+F} add note F to akkord\par
            \InputN{Shift+G} add note G to akkord\par
            \blank[big]

      For entering intervalls there are the following
      commands:

      \blank[medium]
      \startcolumns[n=2]
            \InputN{1}       Prime up\par
            \InputN{2}       Sekunde up\par
            \InputN{3}       Terz up\par
            \InputN{4}       Quarte up\par
            \InputN{5}       Quinte up\par
            \InputN{6}       Septe up\par
            \InputN{7}       Septime up\par
            \InputN{8}       Oktave up\par
            \InputN{9}       None up\par

            \InputN{Shift+1} Prime down\par
            \InputN{Shift+2} Sekunde down\par
            \InputN{Shift+3} Terz down\par
            \InputN{Shift+4} Quarte down\par
            \InputN{Shift+5} Quinte down\par
            \InputN{Shift+6} Septe down\par
            \InputN{Shift+7} Septime down\par
            \InputN{Shift+8} Oktave down\par
            \InputN{Shift+9} None down\par
      \stopcolumns
            \blank[big]

      More note entry commands:

            \blank[medium]
            \InputN{x}   flip note stem direction\par
            \blank[big]

      \subsection{Note entry with mouse}

      In note entry mode a gray note head shows the position 
      a click whould insert a note. Setting a note replaces a
      rest or note. Shift+Click adds a note, building a chord.

      \subsection{Note entry with midi keyboard}
      \subsection{Select}

      \startpacked
            \Input{Note}       click on note head\par
            \Input{Akkord}     double click on note head\par
            \Input{+-Note}     Shift + Click on note head\par
      \stoppacked


      \subsection{Accidentals}

      \M\ sets accidentals depending on pitch, signature
      and already set accidentals in current measure.
      If you select a notehead and change pitch with cursor up/down
      \M\ sets accidentals automatically.

      Another way is to select a note and to select an accidental from
      the toolbox. The selected note gets the selected accidental and
      \M\ changes the pitch automatically. This is the way to edit
      enharmonic exchange (??) or to add a security accidental.

      \Noten{noten2.pdf}

      \subsection{Modify}

      The pitch of an selected mode can be changed by:

      \startpacked
            \Input{Up}
                  increase pitch a half tone\par
            \Input{Ctrl+Up}
                  increase pitch one octave\par
            \Input{Down}
                  decrease pitch a half tone\par
            \Input{Ctrl+Down}
                  decrease pitch one octave\par
            \stoppacked

\section{Slurs}

      Slurs are connected top notes or rest. If a note moves, the
      associated slur moves to. If a slur spans a line or page,
      it is automatically splittet into segments.

      To create a slur, you have to first select a start note.
      Entering ''S'' creates a slur to the next note.
      Double click the slur to enter edit mode. In edit mode
      four control points are shown, which can be moved with
      mouse or keyboard commands.

      \Screen{slur1}{Slur with control points}

            \blank[medium]
            \Input{s}   create a slur to the next note and enter edit mode\par
            \Input{Tab} switch to next control point\par
            \Input{right} move the current control point one space to the right\par
            \Input{left}  move the current control point one space to the left\par
            \Input{up}  move the current control point one space up\par
            \Input{down} move the current control point one space down\par
            \Input{Ctrl+right} move control point $\pm 1/10$ space to the right\par
            \Input{Ctrl+left}  move control point $\pm 1/10$ space to the left\par
            \Input{Ctrl+up}    move control point $\pm 1/10$ space up\par
            \Input{Ctrl+down}  move control point $\pm 1/10$ space down\par
            \Input{Shift+right} move control point to the next note or rest\par
            \Input{Shift+left} move control point to previous note or rest\par
            \Input{x}  flip slur orientation\par
            \Input{ESC} leave edit mode\par
            \Input{Doppelclick} start edit mode\par
            \blank[big]


\section{Instruments}

      \Screen{instruments}{Instrument Dialog}

\section{Text}
      \section{Fingering}

            - open fingering palette create->text->fingering
            - click on "finger" and then on note head to put finger number
              to note

            - doubleclick to edit number

      \subsection{Liedtext}

            First select a note or rest were you want to start lyrics entry.

            Ctrl+L      start lyrics entry; a blinking text cursor
                        appears beneath the not

            Tab         positions the cursor to the next note

            Return      creates another lyrics line

            ESC         exit lyrics entry


\section{Beams}

            Staff Crossing Beams

            Shift+Ctrl+Down  move note/chord a staff down in a multi staff
                  Instrument (piano)

            Shift+Ctrl+Up  move note/chord a staff up in a multi staff
                  Instrument (piano)

            x  flips beam orientation above, below; this overrides automatic

\section{Symbols}

      \Screen{symbols}{Feta-Symbols}

\section{Attributes}
      \subsection{Unvisible}
            Most objects on the canvas can have the ''invisible'' attribute.
            Invisible objects use space in the layout but are invisible on
            screen and on printout.
            To be able to switch them back to ''visible'',there is a global
            mode ''show invisible''. In this mode all invisible objects are
            shown grayed out. They are still not visible in printout.

      \subsection{Color}
            For most objects on the canvas you can select a color. The
            objects are shown and printed in this color.

\section{Navigation}

      \startpacked
            \Input{Right}
                  go to next note\par
            \Input{Left}
                  go to previous note\par
            \Input{Alt+Up}
                  go to next higher note in chord or lowest note in
                  higer line\par
            \Input{Alt+Down}
                  select next lower note in chord or highest note in 
                  lower line\par
            \Input{Alt+Ctrl+Up}
                  select highest note in chord\par
            \Input{Alt+Ctrl+Down}
                  slect lowest note in chord\par
            \Input{Drag Canvas}
                  move note sheet on canvas\par
      \stoppacked

      \subsection{Zoom}

      The note sheet can be resized with this methods:

      \startitemize[packed]
            \item after selecting the zoom tool (magnifying glass), the
                  note sheet can be zoomed in with the left mouse key and
                  zoomed out with the right mouse key. 
                  When you hold the keyboard shift key while clicking
                  in zoom mode, the zoom tool stays selected.
            \item in the zoom pulldown menu you can select a zoom factor
      \stopitemize


\section{Palettes}
      To insert a palette object:

       \startitemize[packed]
            \item click on the palette object to select it. The cursor changes
               (arrow up).
            \item click on the note sheet to insert the palette object.
       \stopitemize


   \subsection{Copy + Paste}
      A fast method to copy objects:

      \startitemize[packed]
            \item select an object
            \item click with middle mouse key on the destination location to
               insert a object copy.
      \stopitemize

\section{Layout}
      \subsection{Page layout}
      \subsubsection{Page settings}

            \Screen{pagesettings}{Page settings}


      \subsection{Layout}
      \subsection{Spacing}

\section{Styles}

      \Screen{style1}{Style Editor}
      \Screen{textstyle}{Text Style Editor: Styles}
      \Screen{textstyle1}{Text Style Editor: Positions}

\section{\M\ File format}

      \M\ saves a score in file with suffix
      \type{*.msc}. This is an normal text file in XML format.
      The file can be edited with a normal text editor.
      There is no formal description of the \M\ file format
      (i.e. a DTD). The file format will change in the future
      (until there is a 1.0 release) and you should use
      MusicXML to archive scores.


\chapter{Installation}

      \section{Download}
            The latest \M\ Version is available at
            \type{http://mscore.sourceforge.net}

            \M\ is distributed as a compressed TAR file. After download it can
            be expanded with:\par

            \Cmd{tar xvofj mscore-1.0.0.tar.bz2}

            This command creates a subdirectory
            mscore-1.0.0, with the expanded \M\ source.

   \section{Requirements}
      Several linux distributions are splitting packets in an user and an
      developer part. To create \M\ you need both parts.

            \startitemize[packed]
                  \item qt4 gui library in der Version $\ge$ 4.0.0
                  \item ALSA version 0.9.0 or newer for midi input
                  \item JACK audio server
                  \item fluidsynth-1.0.0 and a suitable sound font
                  \item gcc 3.4 or newer
                  \item recommended: kde3 (\M\ uses the KDE printer by default.
                        Without KDE you have to configure your own print command.
            \stopitemize

   \section{Compile}
      Compiliation of \M\ is done the standard way with \type{configure}, then
      \type{make} followed by \type{make install}. For last command you usually
      need to be super user.

   \subsection{Configure}
      Hint: configure works only from a X11 console.

      \CCmd{cd mscore-x.x.x}{./configure --prefix=/usr}

      (if configure does not find the qt libraries you can enter something
      like:  configure --with-qt-prefix=/usr/lib/qt3
      which is reported to work with Mandrake)

      If you have installed GNU Compiler $\ge$ 3.4, then the translatin can be
      speed up by using precompiled headers. For this you have to configure
      like:

      \Cmd{./configure --enable-pch}

   \subsection{Make}

      \Cmd{make}

      compiles the sources and produces the executable file
      \type{mscore}.

      \FrameTitle{Hint}
      \StartFrame
      between different C++ Compiler there are small differences in the
      ABI (Application Binary Interface) which may lead to trouble.
      \M\ should be compiled with the same compiler as your Qt-Library.
      \StopFrame

   \section{Installation}

      \Cmd{su -c make install}

      this installs \M.

      Unfortunately the needed score font is not automatically installed.
      You have to do it by hand.

      You dont need to install the score font for printing. \M\ embedds the
      font data into the printer data stream.

\chapter{Keyboard Shortcuts}
      \Sc{\Uparrow} Shift
      \blank[line]

      \startcolumns[n=2,rule=on]
      \Sc{Alt+F} File Menu\par
      \Sc{Alt+E} Edit Menu\par
      \Sc{Alt+C} Create Menu\par
      \Sc{Alt+N} Notes Menu\par
      \Sc{Alt+L} Layout Menu\par
      \Sc{Alt+S} Style Menu\par
      \Sc{Alt+D} Display Menu\par
      \Sc{Alt+H} Help Menu\par
      \blank[small]
      \hrule
      \blank[small]
      \Sc{A}     Note a\par
      \Sc{B}     Note b\par
      \Sc{C}     Note c\par
      \Sc{D}     Note d\par
      \Sc{E}     Note e\par
      \Sc{F}     Note f\par
      \Sc{G}     Note g\par
      \blank[small]
      \hrule
      \blank[small]
      \Sc{\Uparrow A} add note a\par
      \Sc{\Uparrow B} add note b\par
      \Sc{\Uparrow C} add note c\par
      \Sc{\Uparrow D} add note d\par
      \Sc{\Uparrow E} add note e\par
      \Sc{\Uparrow F} add note f\par
      \Sc{\Uparrow G} add note g\par
      \blank[small]
      \hrule
      \blank[small]
      \Sc{1}  Prime up\par
      \Sc{2}  Sekunde up\par
      \Sc{3}  Terz up\par
      \Sc{4}  Quarte up\par
      \Sc{5}  Quinte up\par
      \Sc{6}  Septe up\par
      \Sc{7}  Septime up\par
      \Sc{8}  Oktave up\par
      \Sc{9}  None up\par
      \blank[small]
      \hrule
      \blank[small]
      \Sc{\Uparrow 1} Prime down\par
      \Sc{\Uparrow 2} Sekunde dwon\par
      \Sc{\Uparrow 3} Terz down\par
      \Sc{\Uparrow 4} Quarte down\par
      \Sc{\Uparrow 5} Quinte down\par
      \Sc{\Uparrow 6} Septe down\par
      \Sc{\Uparrow 7} Septime down\par
      \Sc{\Uparrow 8} Oktave down\par
      \Sc{\Uparrow 9} None down\par
      \blank[small]
      \hrule
      \blank[small]
      \Sc{Right}  next note\par
      \Sc{Left}   previous note\par
      \Sc{Alt+Up} select higher note in chord\par
      \Sc{Alt+Down} select lower note in chord\par
      \Sc{Alt+Ctrl+Up}  select highest nost in chord\par
      \Sc{Alt+Ctrl+Down} select lowest note in chord\par

      \stopcolumns

\chapter{Design \& Implementation}

   \section{Main Structure}

      \M\ handles multiple documents (scores). The documents can
      selected with a tab bar. A global pointer points to the current
      selected score (\type{cs}).

      A \type{Score} contains data read from an \type{*.msc} or \type{*.xml}
      file. Think of it as a simple list of measures. The \type{layout()}
      procedure breaks this list into lines ans pages producing a
      \type{Layout} structure. The \type{Layout} can be viewed by a
      \type{Viewer} providing a magnification and horizontal und
      vertical offsets.

      \setupFLOWcharts
        [dx=2\bodyfontsize,
         dy=2\bodyfontsize,
         width=7\bodyfontsize,
         height=3\bodyfontsize,
         maxwidth=\textwidth]

      \startFLOWchart[mops1]

        \startFLOWcell
          \name{MuseScore}
          \location{1,1}
          \shape{procedure}
          \text{MuseScore}
          \connect[bl]{Project1}
          \connect[bl]{Project2}
          \stopFLOWcell

        \startFLOWcell
          \name{Project1}
          \location{1,2}
          \shape{32}
          \text{Score\\A}
          \connect[rl]{Layout1}
          \connect[rl]{Layout2}
          \stopFLOWcell

        \startFLOWcell
          \name{Layout1}
          \location{2,2}
          \shape{34}
          \text{Layout}
          \connect[rl]{View1}
          \stopFLOWcell

        \startFLOWcell
          \name{View1}
          \location{3,2}
          \shape{17}
          \text{View PagePreView}
          \stopFLOWcell

        \startFLOWcell
          \name{Layout2}
          \location{2,3}
          \shape{34}
          \text{Layout}
          \connect[rl]{View2}
          \connect[rl]{View3}
          \stopFLOWcell

        \startFLOWcell
          \name{View2}
          \location{3,3}
          \shape{17}
          \text{View\\Canvas}
          \stopFLOWcell

        \startFLOWcell
          \name{View3}
          \location{3,4}
          \shape{17}
          \text{View Navigator}
          \stopFLOWcell

        \startFLOWcell
          \name{Project2}
          \location{1,3}
          \shape{32}
          \text{Score\\B}
          \stopFLOWcell

      \stopFLOWchart
      \FLOWchart[mops1]

      A \type{Score} can have more than one \type{Layout}. This is used in
      the \type{Page Settings} dialog for a page preview.

      A \type{Layout} can have more than one \type{View}. The main view
      is called \type{Canvas} and can be used to view and edit the score.
      A second smaller \type{View} ist used by the \type{Navigator} widget.

   \section{Layout Object Hierarchy}

      \setupFLOWcharts
        [dx=1.5\bodyfontsize,
         dy=1.5\bodyfontsize,
         width=8\bodyfontsize,
         height=4\bodyfontsize,
         maxwidth=\textwidth]

      \startFLOWchart[mops2]

        \startFLOWcell
          \name{Page}
          \location{1,1}
          \shape{procedure}
          \text{Page}
          \connect[rl]{System}
          \stopFLOWcell

        \startFLOWcell
          \name{System}
          \location{2,2}
          \shape{procedure}
          \text{System}
          \connect[rl]{Measure}
          \stopFLOWcell

        \startFLOWcell
          \name{Measure}
          \location{3,3}
          \shape{procedure}
          \text{Measure}
          \connect[rl]{Slur}
          \connect[rl]{SlurSegment}
          \connect[rl]{Segment}
          \stopFLOWcell

        \startFLOWcell
          \name{Slur}
          \location{4,4}
          \shape{procedure}
          \text{Slur}
          \stopFLOWcell

        \startFLOWcell
          \name{SlurSegment}
          \location{4,5}
          \shape{procedure}
          \text{SlurSegment}
          \stopFLOWcell

        \startFLOWcell
          \name{Segment}
          \location{4,6}
          \shape{procedure}
          \text{Segment}
          \connect[rl]{Chord}
          \connect[rl]{Rest}
          \connect[rl]{Lyric}
          \stopFLOWcell

          \startFLOWcell
            \name{Chord}
            \location{5,7}
            \shape{procedure}
            \text{Chord}
            \connect[rl]{Note}
            \stopFLOWcell

            \startFLOWcell
              \name{Note}
              \location{6,8}
              \shape{procedure}
              \text{Note}
              \connect[rl]{Accidental}
              \stopFLOWcell

              \startFLOWcell
                \name{Accidental}
                \location{7,9}
                \shape{procedure}
                \text{Accidental}
                \stopFLOWcell

          \startFLOWcell
            \name{Rest}
            \location{5,10}
            \shape{procedure}
            \text{Rest}
            \stopFLOWcell

          \startFLOWcell
            \name{Lyric}
            \location{5,11}
            \shape{procedure}
            \text{Lyric}
            \stopFLOWcell

      \stopFLOWchart
      \FLOWchart[mops2]

\chapter{Examples}
      \vbox{\placecontent}
      \setuphead[section][placehead=no]

      \definemakeup[examples][
         doublesided=yes,page=yes,pagestate=start
         ]

      \startexamplesmakeup
      \section{Bilder einer Ausstellung -- Promenade, Modeste Mussorgsky}
      \externalfigure[pics/promenade.pdf][width=1.1\textwidth]
      \stopexamplesmakeup

      \startexamplesmakeup
      \section{Invention No1, J.S.Bach}
      \externalfigure[pics/inv1-1.pdf][width=1.1\textwidth]
      \stopexamplesmakeup

      \startexamplesmakeup
      \externalfigure[pics/inv1-2.pdf][width=1.1\textwidth]
      \stopexamplesmakeup

      \startexamplesmakeup
      \section{Invention No6, J.S.Bach}
      \externalfigure[pics/inv6-1.pdf][width=1.1\textwidth]
      \stopexamplesmakeup

      \startexamplesmakeup
      \externalfigure[pics/inv6-2.pdf][width=1.1\textwidth]
      \stopexamplesmakeup

\completeindex
\stoptext
