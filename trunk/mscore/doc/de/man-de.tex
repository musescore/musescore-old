%====================================================================
%     MuseScore Manual
%     Dies ist das MuseScore Handbuch
%
% (C) 2005-2007 Copyright: Werner Schweer
%====================================================================

%----------------------------------------------------------
%     Setup
%----------------------------------------------------------

%\showframe
%\showsetups
\usemodule[chart]
\setupcolors[state=start]
\setupbodyfont[Palatino]
\language[de]
% typeset in unicode (utf8)
\enableregime[utf]

\startuseMPgraphic{FunnyFrame}
      picture p; numeric o; path a, b; pair c;
      p := textext.rt(\MPstring{FunnyFrame});
      o := BodyFontSize;
      a := unitsquare xyscaled(OverlayWidth,OverlayHeight);
      p := p shifted (20, OverlayHeight-ypart center p);
      drawoptions (withpen pencircle scaled 1pt withcolor .625red);
      b := a superellipsed .95;
      fill b withcolor .85white; draw b;
      b := (boundingbox p) superellipsed .95;
      fill b withcolor .85white; draw b;
      draw p withcolor black;
      setbounds currentpicture to a;
    \stopuseMPgraphic

\defineoverlay[FunnyFrame][\useMPgraphic{FunnyFrame}]
\defineframedtext[FunnyText][frame=off,background=FunnyFrame]
\def\StartFrame{\startFunnyText}
\def\StopFrame{\stopFunnyText}
\def\FrameTitle#1%
 {\setMPtext{FunnyFrame}{\hbox spread 1em{\hss\strut#1\hss}}}
\setMPtext{FunnyFrame}{}


\define\M{Muse\-Score}
\define[1]\Index{{\it #1}\marginpar{#1}\index{#1}}

\defineindenting[Cmdi][text=,separator=,width=fit,distance=1em]

%\define\startdescription{\startpacked}
\define\startdescription{}
%\define\stopdescription{\stoppacked}
\define\stopdescription{}

\define[1]\Fig{\hbox{\externalfigure[#1]}}

\define[2]\Screen{
      \placefigure[here][fig:#1]{#2}{\externalfigure[#1][scale=1600]}
      }
\define[1]\Noten{
      \blank[line]
      {\externalfigure[#1][scale=700]}
      \blank[line]
      }

\definedescription[Option][
      location=left,
      headstyle=bold,
      width=4em,
      before={\startnarrower[left]\setupblank[0pt]},
      after={\stopnarrower\setupblank}]

\definedescription[Sc][
      location=left,
      headstyle=bold,
      width=4em,
      before={\startnarrower[left]\setupblank[0pt]},
      after={\stopnarrower\setupblank}]

\definedescription[Filetype][
      location=left,
      headstyle=bold,
      width=4em,
      before={\startnarrower[left]\setupblank[0pt]},
      after={\stopnarrower\setupblank}]

\definedescription[FileList][
      location=left,
      headstyle=bold,
      width=7em,
      before={\startnarrower[left]\setupblank[0pt]},
      after={\stopnarrower\setupblank}]

\definedescription[Input][
      location=left,
      headstyle=bold,
      width=9em,
      before={\startnarrower[left]\setupblank[0pt]},
      after={\stopnarrower\setupblank}]

\definedescription[InputN][
      location=left,
      headstyle=bold,
      width=6em,
      before={\startnarrower[left]\setupblank[0pt]},
      after={\stopnarrower\setupblank}]

\define[2]\Figure{
      \placefigure
         [#1][fig:#2]{}
         {\externalfigure[#2]}
         }

\define[1]\Menu{
      \placefigure[right][fig:#1]{}
         {\externalfigure[#1][scale=2000]}
         }

\definedescription[Opt][location=hanging,headstyle=bold,width=broad]
\setuptyping[before=\blank\startbackground, after=\stopbackground\blank] % source code with background

\component figurepath.tex

%----------------------------------------------------------
%     Body
%----------------------------------------------------------

\starttext
   \language[de]
   \mainlanguage[de]
   \startstandardmakeup[doublesided=no]
      \definebodyfont[10pt,11pt,12pt][rm][tfe=Regular at 48pt]
      \tfe\setupinterlinespace
      \hfill \color[red]{\M}\par
      \hfill \color[blue]{Manual}\par
      \vfill
      \hfill \color[blue]{DE}\par
      \definebodyfont[10pt,11pt,12pt][rm][tfb=Regular at 24pt]
      \tfb\setupinterlinespace
      \hfill Version 0.4\par
      \stopstandardmakeup

   \startstandardmakeup[page=no]
      \vfill
      Das \M\ Projekt ist auf SourceForge gehostet:\par
      \type{http://mscore.sourceforge.net/}
      \blank[line]
      \M\ verwendet die Fonts des Lilypond Typesetter Projekts.\par
      Dieses Dokument wurde mit \pdfTeX\ und dem Macropaket \ConTeXt\ gesetzt.
      \blank[line]
      \copyright 2005-2007 Werner Schweer und Andere
      \stopstandardmakeup

   \completecontent

\chapter{Einführung}
   \section{Was ist \M?}
      \M\ ist ein WYSIWYG (What You See Is What You Get)
      Programm zum Setzen und Drucken von Noten.

      \M\ ist ''Open Source'' Software und wird unter der GNU General
      Public License (GPL) veröffentlicht (siehe Anhang).

      Einige Highlights:

      \startitemize[packed]
            \item WYSIWYG Design, Noten können auf einem "virtuellen Notenblatt" eingegeben
                  werden.
            \item \M\ verwendet TrueType Font(s) zum Drucken und für die
                  Bildschirmausgabe. Dies ermöglicht eine hochwertige Darstellung
                  in allen Vergrößerungsstufen.
            \item Noten können einfach und schnell auch nur über die Tastatur
                  eingegeben werden.
            \item MusicXML Import/Export
            \item Midi Import/Export
            \item Midi Input zur Noteneingabe
            \item Integrierter Sequenzer und Software Synthesizer zum Vorspielen
                  der Partitur.
      \stopitemize

   \section{Kurzanleitung}
      \subsection{Programmstart}

      \Screen{main1}{Leeres Hauptfenster}

      Nach der Eingabe von

      \starttyping
mscore
      \stoptyping

      wird das zuletzt bearbeitete Projekt gezeigt. Beim ersten
      Start von \M\ erscheint das leere Hauptfenster (Abb. \in[fig:main1]).

   Das Fenster besteht aus drei Bereichen:
   \startitemize[packed]
      \item einer Leiste mit Pulldownmenü
      \item eine Leiste mit Werkzeugsymbolen
      \item einem leeren Canvas
   \stopitemize

  \subsection{Neue Partitur erzeugen}

      Um ein Notenblatt zu erzeugen geben Sie \type{Strg+N} ein.
      Es erscheint ein Dialog mit einer Liste von Vorlagen (''Templates'') zur Auswahl
      (Abb. \in[fig:template]).
      \index{Vorlagen}\index{Templates}

      \Screen{template}{Template Auswahl}

      Wir wählen \type{a4piano.msc} und verlassen den Dialog
      mit \type{OK}.

      Auf dem dem Canvas des Hauptfensters wird nun ein Arbeitsblatt
      mit dem gewünschten Template angezeigt, in unserem Fall also zwei
      Notenzeilen, die mit einer Klammer verbunden sind. Die Noteschlüssel
      sind gesetzt, die Taktart ist mit $4/4$ vorgegeben und die Notenzeilen
      sind bereits mit ein paar Pausentakten bevölkert.

      \Screen{emptypiano}{Leeres Pianosystem}

\subsection{Noteneingabe}

      Nun wollen wir ein paar Noten eingeben. Dazu clicken wir zunächst die
      erste Pause an um sie zu selektieren und starten dann
      den Noteneingabemode durch Eingabe von \type{N}.
      Der Cursor verändert sich, um den Eingabemodus anzuzeigen.
      Ebenso erscheint eine Einfügemarke.

      Um die leeren Takte mit Noten zu bevölkern, tippen wir nun
      einfach ''c d e'' auf der Tastatur ein. Es ergibt sich
      folgendes Bild:

      \Noten{noten1.pdf}

\subsection{Editieren}
   Einige Aktionen:
   \startitemize[packed]
      \item mit der gedrückten linken Maustaste läßt sich das Notenblatt
            auf dem Canvas verschieben.
      \item durch tippen von \type{Strg+M} können weitere leere Takte
            angehängt werden
      \item mit der Lupe aus der Werkzeugleiste läßt sich die Vergrößerung
           des  Notenblattes verändern
   \stopitemize

\subsection{Drucken}

      Durch anclicken des Druckerknopfes wird der Druckdialog
      gestartet. \M\ erzeugt PostScript oder PDF Druckerdaten,
      die zum Drucker geschickt oder in eine Datei umgeleitet werden
      können.

      \FrameTitle{Tip:}
      \StartFrame
      Um Notenbeispiele für ein Buch zu erzeugen, erstellen sie mit
      \M\ eine kleine Partiturseite im A5 Querformat (Landscape Option).
      Nach Ausgabe in eine PDF-Datei
      wird dann überflüssiger Rand mit dem Zeilenkommando \type{pdfcrop}
      abgeschnitten. Das Resultat kann in ein \ConTeXt\ Dokument
      integriert werden. Die Beispiele in diesem Handbuch wurden auf
      diese Weise erstellt.
      \StopFrame

%----------------------------------------------------------
%
%     Referenz
%
%----------------------------------------------------------

\chapter{Referenz}
   \section{Programmstart}

      \M\ kann von der Kommandozeile aus durch tippen von
      \index{Kommandozeile}

\starttyping
mscore <Optionen> <Dateiname>
\stoptyping

      gestartete werden. \type{Optionen} und \type{Dateiname} sind
      optional.
      \index{Aufrufoptionen}\index{Optionen}\index{Versionsnummer}
      \index{Debug-Mode}

      Es gibt folgende Optionen:
            \startnarrower[left]
            \starttabulate
            \NC -v \NC gibt die aktuelle Versionsnummer von \M\ aus\NC\NR
            \NC -d \NC startet \M\ im Debug-Modus\NC\NR
            \NC -s \NC startet \M\ ohne integrierten Software Synthesizer\NC\NR
            \NC -m \NC startete \M\ ohne MIDI Input Support\NC\NR
            \stoptabulate
            \stopnarrower

      Die Optionen der aktuellen Version können durch Eingabe der
      Option \type{-h} (''Hilfe'') erfragt werden.
      \blank[line]

      Als Datei akzeptiert \M\ folgenden Typen:
            \startnarrower[left]
            \starttabulate
            \NC *.msc \NC \M\ Partiturdatei\NC\NR
            \NC *.mss \NC \M\ Style\NC\NR
            \NC *.xml \NC MusicXML Datei\NC\NR
            \NC *.mid \NC MIDI--Datei\NC\NR
            \stoptabulate
            \stopnarrower
      \index{Dateitypen}

      Wird \M\ ohne Angabe einer Datei gestartet, dann wird je nach
      Voreinstellung die zeletzt bearbeitete Partitur, eine leere Partitur
      oder eine feste voreingestellte Partitur gezeigt. Beim allerersten Start von
      \M\ wird als voreingestellte Partitur ein Demofile gezeigt.

      \M\ schreibt/liest im Hintergrund zwei weitere Dateien:
            \blank[medium]
            \startnarrower[left]
            \starttabulate
            \NC \type{~/.mscore}    \NC \M\ Voreinstellungen \NC\NR
            \NC \type{~/.mscorePrj} \NC enthält die zuletzt bearbeiteten Partituren\NC\NR
            \stoptabulate
            \stopnarrower
            \blank[big]


   \section{Menü}
      Viele Menüoptionen stehen auch in Form von Buttons in einer
      Toolbox zur Verfügung.

      Menü und Tastenkürzel:

            \blank[medium]
            \startnarrower[left]
            \starttabulate
            \NC \type{Alt+F} \NC Datei\NC\NR
            \NC \type{Alt+E} \NC Bearbeiten\NC\NR
            \NC \type{Alt+C} \NC Erzeugen\NC\NR
            \NC \type{Alt+N} \NC Noten\NC\NR
            \NC \type{Alt+L} \NC Layout\NC\NR
            \NC \type{Alt+S} \NC Stil\NC\NR
            \NC \type{Alt+D} \NC Anzeige\NC\NR
            \NC \type{Alt+H} \NC Hilfe\NC\NR
            \stoptabulate
            \stopnarrower
            \blank[big]

      \subsection{File Menü}
      \Menu{menu1}

      \startpacked
      \Opt{New (Strg+N)} erzeugt ein neues Notenblatt. Als
            nächstes müssen zunächst Instrumente und dann Leertakte
            erzeugt werden.\par
      \Opt{Open} öffnet eine neue Partitur\par
      \Opt{Open Recent} zeigt eine Liste der zuletzt bearbeiteten
        Partituren. Ein Click auf eine Partitur öffnet sie zur
        Bearbeitung.\par
      \Opt{Save} schreibt die aktuelle Partitur auf die
        Festplatte\par
      \Opt{Save As} schreibt die aktuelle Partitur unter einem
        zu wählenden Namen auf die Festplatte\par

      \Opt{Export Midi} Exportiert die aktuelle Partitur als
        MIDI-Datei. \index{Midi-Datei}

      \Opt{Export MusicXML} Exportiert die aktuelle Partitur als
        MusicXML Datei. \index{MusicXML}

      \Opt{Import Midi} Import eine MIDI-Datei und versucht, sie
        als Partitur darzustellen.

      \Opt{Import MusicXML} Importiert eine MusicXML Datei.

      \Opt{Print (Strg+P)} druckt die aktuelle Partitur.
        \index{Drucken}

      \Opt{Quit} Beendet \M.\par
        \index{Beenden}

      \stoppacked


      \subsection{Edit Menu}

      \Menu{menu2}

      \startpacked
      \Opt{Undo} der letzte Bearbeitungsschritt wird rück\-gäng\-ig
                  gemacht. Es können beliebig viele Bearbeitungsschritte
                  rückgängig gemacht werden.

      \Opt{Redo}  ''Undo'' den letzte Undo Schritt.

      \Opt{Cut}

      \Opt{Copy}

      \Opt{Paste}

      \Opt{Instrument List}   zeigt die Instrument Liste, in der
                  wei\-tere Einstellungen zu den Instrumenten gemacht
                  werden können.

      \Opt{Page List}   Zeigt die Page List. Dies ist eine Debug
                  Option, um die internen Datenstrukturen von \M\
                  anzuzeigen.

      \Opt{Preferences} Zeigt den Preference Dialog, in dem zahlreiche
            Voreinstellungen gemacht werden können.\par

      \stoppacked



      \subsection{Create Menu}

            \Menu{menu3}

      \startpacked
            \Opt{Instruments}
                  Öffnet den Instrumenten-Dialog. Dort können
                  Instrumente hinzugefügt und entfernt werden. Ebenso
                  können weitere Notenlinien einem Instrument zugefügt
                  werden.\par
            \Opt{Measure}
                  fügt einen leeren Takt an die Partitur an. Der Takt
                  ist mit einer Pause gefüllt.\par
            \Opt{Clef}
                  Öffnet die Notenschlüssel Palette.\par
            \Opt{Key}
                  Öffnet die Vorzeichen Palette.\par
            \Opt{Time}
                  Öffnet die Taktart Palette.\par
            \Opt{Lines}
                  Öffnet die Linien Palette (crescendo etc.)\par
            \Opt{Note Attributes}
                  Öffnet die Noten Attribute Palette.\par
            \Opt{Dynamics}
                  Öffnet die Dynamics Palette.\par
            \Opt{Text}
                  Öffnet das Text Submenü
            \Opt{Symbols}
                  Öffnet die Symbol Palette.\par

            \stoppacked



      \subsection{Notes Menu}

            \Menu{menu4}

         \startpacked

            \Opt{Input}
                  schaltet in den Noteneingabemodus um.
                  \index{Noteneingabemodus}\par
            \Opt{Add Pitch}
                  zeigt ein weiteres Untermenü mit Notenwerten,
                  die Eigegeben werden können.\par
            \Opt{Add Intervall}
                  zeigt eine Liste von Intervallen zum Erzeugen
                  von Akkorden.\par
            \Opt{N-Tole}
                  zeigt ein Untermenü zur Erzeugung von Triolen,
                  Pentolen usw.\par
            \stoppacked

         \startcombination[3*1]
            {\externalfigure[menu41][scale=1200]}{}
            {\externalfigure[menu42][scale=1200]}{}
            {\externalfigure[menu43][scale=1200]}{}
            \stopcombination

      \subsection{Layout Menu}

            \Menu{menu5}

      \startpacked
            \Opt{PageSettings}
                  zeigt den Seiteneinstellungs Dialog.\par
            \Opt{Reset Positions}
                  stellt alle markierten Objekte wieder auf ihre
                  Standardposition zurück. Dies macht ein manuelles
                  Verschieben von Objekten rückgängig. Ebenso werden
                  manuell veränderte Notenhalsrichtugen wieder auf
                  ihre Standardausrichtung zu\-rück-ge\-stellt.\par
            \Opt{Set Normal Staff Distances} stellt manuell verkürzte
                  Notenlinienabstände wieder auf ihre Normalwerte
                  zurück\par
            \Opt{Reset Stretch}
                  Setzt gedehnte Takte wieder auf ihre normale Breite
                  zu\-rück\par
            \Opt{System Break}
                  Erzeugt hinter dem markierten Takt einen Zeilenumbruch.\par
            \Opt{Page Break}
                  Erzeugt hinter dem markierten Takt einen
                  Seitenumbruch.\par
            \stoppacked


      \subsection{Style Menu}

            \Menu{menu6}

      \startpacked
            \Opt{Edit Style}
                  zeigt den Style Editor.\par
            \Opt{Edt Text Style}
                  zeigt den Text Style Editor.\par
            \Opt{Load Style}
                  Läd einen neuen Style aus einer Datei.\par
            \Opt{Save Style}
                  Schreibt den aktuellen Style in eine Datei.\par
            \stoppacked


      \subsection{Display Menu}

            \Menu{menu7}

      \startpacked
            \Opt{Pad}
                  schaltet das Eigabe-Pad ein/aus.\par
            \Opt{Play Panel}
                  schaltet das Play Panel ein/aus.\par
             \Opt{Transport Toolbar}
                  schaltet die Transport Werkzeugleiste ein/aus.\par
            \Opt{Show Invisible}
                  Nach dem Einschalten werden unsichtbar geschaltete
                  Notenelemente auf der Partitur in grau dargestellt.
                  Dies ermöglicht ihre Bearbeitung.\par

            \stoppacked

      \subsection{Help Menu}

            \Menu{menu8}

      \startpacked
            \Opt{Browser}
                  startete den konfigurierten Browser mit der Online-Hilfe.\par
            \Opt{About}
                  zeigt das ''About'' Panel.\par
             \Opt{About Qt}
                  zeigt Informationen über das GUI-Toolkit Qt der Firma
                  Trolltech, mit dem \M\ erstellt wurde.\par
            \Opt{Whats This}
                  nach dem Anclicken erscheint der ''whats this'' Cursor mit
                  dem irgendein GUI-Element angeclickt werden kann. Soweit
                  implementiert wird dann ein Hilfetext zu diesem Element
                  gezeigt.\par
            \stoppacked

%----------------------------------------------------------
%     Noteneingabe
%----------------------------------------------------------

   \section{Noten Eingabe}

      \M\ befindet sich immer in einem von zwei Modi:

      \startitemize[packed]
            \item {\bf Edit-Mode}, (normal) wird durch einen normalen
                  Pfeilcursor angezeigt.
            \item {\bf Eingabemode}, wird durch einen nach oben zeigenden
                  Pfeilcursor angezeigt.
      \stopitemize

            \blank[medium]
            \Input{N} Noten-Eingabemodus starten\par
            \Input{Esc} Eingabemodus beenden\par
            \blank[big]

      \FrameTitle{Hinweis:}
      \StartFrame
      Im Eingabemodus beziehen sich Einstellungen des Pads oder im Toolbar
      immer auf die nächste einzugebende Note, ansonsten wirken Pad-Eingaben
      auf die aktuell selektierte Note.
      \StopFrame

      Der Eigabemode wird durch anclicken einer Note im Pad oder durch
      Eingabe von \type{N} aktiv.

      Im Eingabemodus wird beim clicken auf die Partitur immer ein
      Objekt gesetzt. Der Eingabemode wird durch anclicken des
      Pfeils im Pad oder durch Eingabe von \type{ESC} verlassen.

      \subsection{Tastatureingaben}

      Noten können mit
            \blank[medium]
            \Input{a b c d e f g} eingegeben werden\par
            \blank[big]

      eingegeben werden. Die Noten werden an der aktuellen Position
      eingegeben. Die aktuelle Position kann durch anclicken einer
      Note oder einer Pause gesetzt werden und wird durch einen
      blinkenden Cursor angezeigt. Der Cursor steht immer vor der
      aktuellen Position. Die Länge der aktuellen Note sowie andere
      Eigenschaften können mit dem Pad gesetzt werden. Das Pad
      visualisiert die Tasten des numerischen Keypads einer normalen
      PC Tastatur.

      Wird eine Note selektiert, dann können mit den nächsten
      Kommandos Akkorde erstellt werden. Jede Eingabe schaltet
      zudem in den Noteneingabemodus:

      \startnarrower[left]
      \starttabulate[|l|l|]
      \NC\bf Shift+A\NC  fügt Note A zum Akkord hinzu\NC\MR
      \NC\bf Shift+B\NC  fügt Note B zum Akkord hinzu\NC\MR
      \NC\bf Shift+C\NC  fügt Note C zum Akkord hinzu\NC\MR
      \NC\bf Shift+D\NC  fügt Note D zum Akkord hinzu\NC\MR
      \NC\bf Shift+E\NC  fügt Note E zum Akkord hinzu\NC\MR
      \NC\bf Shift+F\NC  fügt Note F zum Akkord hinzu\NC\MR
      \NC\bf Shift+G\NC  fügt Note G zum Akkord hinzu\NC\MR
      \stoptabulate
      \stopnarrower

      Für die Eingabe von Intervallen gibt es folgende
      Kommandos:

      \startnarrower[left]
      \starttabulate[|l|l|l|l|]
      \NC\bf 1 \NC Prime nach oben   \NC\bf Shift+1 \NC Prime nach unten   \NC\MR
      \NC\bf 2 \NC Sekunde nach oben \NC\bf Shift+2 \NC Sekunde nach unten \NC\MR
      \NC\bf 3 \NC Terz nach oben    \NC\bf Shift+3 \NC Terz nach unten    \NC\MR
      \NC\bf 4 \NC Quarte nach oben  \NC\bf Shift+4 \NC Quarte nach unten  \NC\MR
      \NC\bf 5 \NC Quinte nach oben  \NC\bf Shift+5 \NC Quinte nach unten  \NC\MR
      \NC\bf 6 \NC Septe nach oben   \NC\bf Shift+6 \NC Septe nach unten   \NC\MR
      \NC\bf 7 \NC Septime nach oben \NC\bf Shift+7 \NC Septime nach unten \NC\MR
      \NC\bf 8 \NC Oktave nach oben  \NC\bf Shift+8 \NC Oktave nach unten  \NC\MR
      \NC\bf 9 \NC None nach oben    \NC\bf Shift+9 \NC None nach unten    \NC\MR
      \stoptabulate
      \stopnarrower

      Weitere Kommandos zur Noteneingabe:

      \startnarrower[left]
      \starttabulate[|l|p|]
         \NC\bf x                \NC kehrt den Notenhals um\NC\MR
         \NC\bf Shift+Strg+Down  \NC Verschiebt eine Note oder Pause in die nächste
                                    Notenzeile. Dies wird bei der Notation von
                                    Pianomusik häufiger verwendet. Noten aus verschiedenen
                                    Notenzeilen können mit dieser Funktion auch gemeinsam
                                    verbalkt werden.\NC\MR
         \NC\bf Shift+Strg+Up    \NC Verschiebt eine Note oder Pause in die höhere Notenzeile
                                    eines mehrzeiligen Instruments.\NC\MR
      \stoptabulate
      \stopnarrower

      \subsection{Noteneingabe mit Maus}

      Im Noteneingabemodus zeigt ein grauer Notenkopf an, wo die
      nächste Note durch Click gesetzt werden kann.
      Shift+Click fügt eine Note hinzu, so das ein Akkord entsteht.

      \subsection{Noteneingabe mit Midi-Keyboard}

      Die Midi-Noteneingabe muß zunächst eingeschaltet werden. Im Noteneingabemode
      erzeugt jeder Midi Note-On Event eine neue Note. Werden auf dem Keyboard
      nach der ersten Taste weitere Tasten gedrückt ohne die erste loszulassen,
      dann wird ein Akkord gesetzt. Nach dem Loslassen aller Tasten rückt \M\ zur
      nächsten Eingabeposition vor.

      Noten können nicht in Echtzeit eingespielt werden. Es kann aber eine Aufnahme
      mit einem Midisequenzer erstellt werden um sie anschließend per Midifile
      zu Importieren.

      \subsection{Auswählen}

      \startnarrower[left]
      \starttabulate[|l|p|]
            \NC\bf Note       \NC Click auf Notenkopf\NC\MR
            \NC\bf Akkord     \NC Doppelclick auf Notenkopf\NC\MR
            \NC\bf +-Note     \NC Shift + Click auf Notenkopf\NC\MR
      \stoptabulate
      \stopnarrower


      \subsection{Vorzeichen}

      \M\ setzt Vorzeichen automatisch in Abhängigkeit von Tonhöhe,
      der Tonart (Vorzeichen am Zeilenanfang) und bereits gesetzter
      Vorzeichen des aktuellen Taktes. Selektieren sie einen Notekopf
      und verändern die Tonhöhe in Halbschritten mit Cursor hoch/tief.
      \M\ setzt in diesem Fall die Vorzeichen automatisch.
      Ein anderes Vorgehen ist es, eine Note zu selektieren und dann
      ein Vorzeichen aus der Toolbox anzuclicken. Die gewünschte Note
      bekommt nun das Vorzeichen und \M\ verändert die Tonhöhe der
      Note gemäß dem gewünschten Vorzeichen. Auf diese Art läßt sich
      auch ein Sicherheitsvorzeichen oder eine enharmonische
      Verwechslung setzen.

      \Noten{noten2.pdf}

      \subsection{Bearbeiten}

      Die Tonhöhe einer selektierten Note kann durch folgende
      Eingaben verändert werden:

      \startnarrower[left]
      \starttabulate[|l|p|]
            \NC\bf Up          \NC erhöht Tonhöhe um einen Halbton\NC\MR
            \NC\bf Strg+Up     \NC erhöht den Ton um eine Oktave nach oben\NC\MR
            \NC\bf Down        \NC verringert Tonhöhe um einen Halbton\NC\MR
            \NC\bf Strg+Down   \NC erniedrigt den Ton um eine Oktave\NC\MR
      \stoptabulate
      \stopnarrower

\section{Bindebögen}
      Bindebögen sind mit Noten oder Pausen verbunden. Wenn sich also
      eine Note verschiebt, dann wird auch der zugehörige Bogen
      mitverschoben. Wenn ein Bogen über eine Zeilengrenze oder
      Seitengrenze läuft, dann wird er automatisch in mehrere Segmente
      aufgeteilt.

      Um einen Bindebogen zu erzeugen muß zunächst die Startnote
      selektiert werden. Mit ''S'' wird ein Bindebogen zu nächsten
      Note erzeugt. Ein Doppelclick auf den Bindebogen schaltet in
      einen Edit-Modus. In diesem Modus werden vier Kontrollpunkte
      gezeigt, die mit der Maus oder mit Tastaturkommandos verschoben
      werden können.

      \Screen{slur1}{Bindebogen mit Kontrollpunkten}

      \startnarrower[left]
            \starttabulate[|l|p|]
            \NC\bf s\NC           erzeugt einen Bindebogen zur nächsten Note und wechselt in den Edit-Modus.\NC\MR
            \NC\bf Tab\NC         wechselt zum nächsten Kontrollpunkt.\NC\MR
            \NC\bf right\NC       verschiebt Kontrollpunkt um ein Spatium nach rechts\NC\MR
            \NC\bf left\NC        verschiebt Kontrollpunkt um ein Spatium nach links\NC\MR
            \NC\bf up\NC          verschiebt Kontrollpunkt um ein Spatium nach oben\NC\MR
            \NC\bf down\NC        verschiebt Kontrollpunkt um ein Spatium nach unten\NC\MR
            \NC\bf Strg+right\NC  verschiebt Kontrollpunkt um $\pm 1/10$ Spatium nach rechts\NC\MR
            \NC\bf Strg+left\NC   verschiebt Kontrollpunkt um $\pm 1/10$ Spatium nach links\NC\MR
            \NC\bf Strg+up\NC     verschiebt Kontrollpunkt um $\pm 1/10$ Spatium nach oben\NC\MR
            \NC\bf Strg+down\NC   verschiebt Kontrollpunkt um $\pm 1/10$ Spatium nach unten\NC\MR
            \NC\bf Shift+right\NC verschiebt Kontrollpunkt zur nächsten Note oder Pause\NC\MR
            \NC\bf Shift+left\NC  verschiebt Kontrollpunkt zur vorherigen Note oder Pause\NC\MR
            \NC\bf x\NC           schaltet die Bogenausrichtung um\NC\MR
            \NC\bf ESC\NC         beendet den Bogen Edit-Modus\NC\MR
            \NC\bf Doppelclick\NC startet den Edit-Modus für den angeclickten Bogen\NC\MR
            \stoptabulate
      \stopnarrower

%%
%%---------------------------------------------------------------------------------------
%%

\section{Instrumente}
      Ein Instrument wird normalerweise durch eine einzelne Notenzeile dargestellt.
      Einige Instrumente besitzen jedoch mehrere Notenzeilen wie z.B. das Piano bei dem
      eine Notenzeile mit Baßschlüssel die Noten der linken und eine Notenzeile mit
      Violinschlüssel die Noten der rechten Hand enthält.

      \Screen{instruments}{Instrumente Dialog}

      Mit dem Instrumentendialog können neue Instrumente und damit Notenzeilen erzeugt,
      gelöscht und verändert werden.
      \blank

      \startpacked
            \Opt{Instrument erzeugen}
                  Nach Auswahl eines Instruments in der Instrumentenliste kann das
                  Instrument mit ''Zufügen'' an die Partitur angehängt werden.\par
            \Opt{Instrument löschen}
                  Ein Instrument oder eine einzelne Notenzeile eines Instruments
                  mit mehreren Notenzeilen kann mit dem ''Entfernen'' Button gelöscht
                  werden.\par
            \Opt{Einzelne Notenzeile zufügen}
                  Mit ''oben Hinzufügen'' oder ''unten Hinzufügen'' können einem
                  Instrument weitere Notenzeilen zugefügt werden.\par
            \Opt{Instrumentenreihenfolge ändern}
                  Ein selektiertes Instrument oder eine selektierte Notenzeile eines
                  Instruments kann mit ''Hoch'' und ''Tief'' in der Partitur
                  verschoben werden.\par
            \stoppacked
      \blank

      Alle Änderungen werden erst beim Verlassen des Instrumentendialogs wirksam und
      als eine Aktion betrachtet. Alles kann somit durch eine einzelnes
      ''Undo'' zurückgestellt werden.
      \blank

      Ist die Partitur leer, dann werden die Instrumente erst nach dem Erzeugen eines
      leeren Taktes sichtbar.

\section{Notenattribute}

      \Screen{noteattributes}{Notenattribute}

\section{Text}
      Texte können an verschiedene Objekte der Partitur verankert werden.
      Titel sind z.B. mit der Textseite verankert, auf der der erste Takt der
      Partitur zu sehen ist. Fingersatzzahlen sind mit Notenköpfen verbunden und
      Liedtexte mit der Zeitachse. Dies bewirkt, das nach Veränderungen der Partitur
      und einem neuen Umbruch die Texte wieder das sind, wo sie erwartet werden.

      Die initialen Eigenschaften eines Textes werden durch die Stil-Vorlage
      festgelegt. Dazu gehören:

      \startnarrower[left]
            \starttabulate[|l|p|]
            \NC\bf Fontfamilie     \NC bestimmt den Namen des Fonts fest \NC\MR
            \NC\bf Punktgröße      \NC Maß für die Schriftgröße (72 Punkte entsprechen einem Zoll = 2,54cm) \NC\MR
            \NC\bf kursiv          \NC \NC\MR
            \NC\bf fett            \NC \NC\MR
            \NC\bf unterstrichen   \NC \NC\MR
            \NC\bf Anker           \NC Seite, Zeit, Notenkopf, System\NC\MR
            \NC\bf Ausrichtung     \NC Horizontal: linksbündig, rechtsbündig, zentriert;
                                    Vertikal: oben, unten, zentriert\NC\MR
            \NC\bf Offset          \NC gibt die Verschiebung des Textes zur Normalposition
                                    an \NC\MR
            \NC\bf Offsettyp       \NC der Offset kann in mm, spatien oder relativ zur
                                    Seitengröße in Prozent angegeben werden\NC\MR
            \stoptabulate
      \stopnarrower

      Nach Doppelclick auf einen Text gelangen wir in den Texteingabemodus. Er ist
      am nun sichtbaren Textcursor zu erkennen. Im Texteingabemodus stehen folgende
      Kommandos zur Verfügung:

      \startnarrower[left]
            \starttabulate[|l|p|]
            \NC\bf links   \NC bewegt den Cursor um eine Spalte nach links\NC\MR
            \NC\bf rechts  \NC bewegt den Cursor um eine Spalte nach rechts\NC\MR
            \NC\bf return  \NC erzeugt eine neue Textzeile\NC\MR
            \NC\bf zurück  \NC löscht das Zeichen links vom Cursor\NC\MR
            \NC\bf delete  \NC löscht das Zeichen rechts vom Cursor\NC\MR
            \NC\bf hoch    \NC bewegt den Cursor in die obere Zeile\NC\MR
            \NC\bf tief    \NC bewegt den Cursor in die untere Zeile\NC\MR
            \NC\bf Pos 1   \NC setzt den Cursor vor das erste Zeichen im Text\NC\MR
            \NC\bf Ende    \NC setzt den Cursor hinter das letzte Zeichen im Text\NC\MR
            \NC\bf Strg+B  \NC schaltet Fettschrift ein/aus\NC\MR
            \NC\bf Strg+I  \NC schaltet Kursivschrift ein/aus\NC\MR
            \NC\bf Strg+U  \NC schaltet Unterstreichen ein/aus\NC\MR
            \NC\bf Strg+tief \NC schaltet in den Subscriptmode\NC\MR
            \NC\bf Strg+hoch \NC schaltet in den Superscriptmode\NC\MR
            \NC\bf F2      \NC schaltet die Eingabepalette ein/aus\NC\MR
            \stoptabulate
      \stopnarrower

      Mit Hilfe der Eingabepalette können Sonderzeichen eingegenben werden, die auf
      einer normalen Tastatur nicht vorhanden sind. Dazu zählen auch einige
      musikalische Symbole.

      \Screen{textpalette}{Text Eingabepalette}

      \subsection{Titel, Subtitel Komponist}
            Diese Texte sind mit der Textseite verbunden.

      \subsection{Fingersatz}
            Fingersatzzahlen können an Notenköpfe verankert werden. Die Position
            der Zahlen wird relativ zu der Position der Notenköpfe gespeichert.
            Verändert sich die Lage der Notenköpfe, dann wandern die Fingersatzzahlen
            entsprechend mit.

            \Screen{fingering}{Fingersatzpalette}

            Das Menü Erzeugen--Text--Fingersatz stellt eine Palette zum bequemen
            Eingeben von Fingersätzen zur Verfügung. Die Fingersatzzahlen können
            von der Palette auf Notenköpfe gezogen werden. Es können übrigens
            beliebig viele Fingersatzzahlen mit einem Notekopf verankert werden.
            \blank
            Fingersatzzahlen können nach Doppelclick wie normaler Text
            editiert werden.

      \subsection{Liedtext}

            Um einen Liedtext einzugeben muß zunächst eine Note oder eine Pause
            für den Start ausgewählt werden.

            \startnarrower[left]
            \starttabulate[|l|p|]
            \NC\bf Strg+L      \NC startet die Texteingabe. Ein Textcursor erscheint
                                unter der ausgewählten Note\NC\MR
            \NC\bf Tab         \NC positioniert den Cursor zur nächsten Note
                                   (nächste Silbe)\NC\MR
            \NC\bf Strg+Tab    \NC positioniert den Cursor zur vorherigen Note
                                   (vorherige Silbe)\NC\MR
            \NC\bf --          \NC positioniert den Cursor zur nächsten Silbe
                                   und verbindet beide Silben mit einer Linie\NC\MR
            \NC\bf Return      \NC erzeugt eine weitere Liedtextzeile\NC\MR
            \NC\bf ESC         \NC beendet den Eingabemode\NC\MR
            \stoptabulate
            \stopnarrower

\section{Balken}

            \startnarrower[left]
            \starttabulate[|l|p|]
              \NC\bf x\NC  Vertauscht die Balkenlage oberhalb und unterhalb der Noten.
                           Dies schaltet die automtische Wahl aus.\NC\MR
            \stoptabulate
            \stopnarrower

\section{Symbole}

      \Screen{symbols}{Feta-Symbole}

\section{Element Attribute}
      Fast alle auf dem Notenblatt dargestellten Elemente haben ein Kontextmenü, welches
      nach Rechtsclicken erscheint. Über das Kontextmenü können einige Eigenschaften
      der Elemente eingestellt werden.

      \subsection{Unsichtbar}
            Die meisten Objekte auf dem Canvas lassen sich mit
            dem ''unsichtbar'' Attribut versehen. Sie sind dann
            zwar im Layout vorhanden, werden aber nicht mehr angezeigt
            oder gedruckt. Um sie wieder sichtbar schalten zu können,
            gibt es einen globalen Modus ''zeige Unsichtbare'', in
            dem alle unsichtbaren Elemente grau dargestellt werden.
            Im Druck erscheinen sie jedoch auch in dieser Einstellung
            nicht.

      \subsection{Farbe}
            Für die meisten Objekte auf dem Canvas läßt sich eine
            Farbe festlegen, mit der sie angezeigt und gedruckt
            werden.

\section{Navigation}

      \startpacked
            \Input{Right}
                  geht zur näcsten Noten\par
            \Input{Left}
                  geht eine Note zurück\par
            \Input{Alt+Up}
                  selektiert höheree Note im Akkord oder niedrigste
                  Note der höheren Notenzeile\par
            \Input{Alt+Down}
                  selektiert niedrigere Note im Akkord oder die höchste
                  Note der nächsten Notenzeile.\par
            \Input{Alt+Strg+Up}
                  selektiert die höchste Note im Akkord\par
            \Input{Alt+Strg+Down}
                  selektiert die niedrigste Note im Akkord\par
            \Input{Drag Canvas}
                  verschiebt das Arbeitsblatt auf dem Canvas\par
      \stoppacked

      \subsection{Zoom und Pan}

      Das Arbeitsblatt kann auf verschieden Weise vergrößert
      oder verkleinert werden:

      \startitemize[packed]
            \item Strg+Mausrad zoomt das Notenblatt
            \item Mausrad verschiebt das Notenblatt horizontal
            \item Shift+Mausrad verschiebt das Notenblatt vertikal
            \item nach Auswahl des Zoom-Werkzeugs kann das Arbeitsblatt
                  mit der linken Maustaste vergrößert und mit der rechten
                  Maus\-taste verkleinert werden. Wird beim klicken die
                  Shift-Taste gedrückt gehalten, dann bleibt das
                  Zoom-Werkzeug ausgewählt.
            \item im Zoom-Pulldownmenü kann eine Vergrößerung direkt
                  ausgewählt werden.
      \stopitemize


\section{Layout}
      \subsection{Seitenlayout}
      \subsubsection{Seiteneinstellungen}

            \Screen{pagesettings}{Seiteneinstellungen}

      \subsection{Umbruch}
            Ein Zeilenumbruch oder ein Seitenumbruch kann erzwungen
            werden. Dazu wird das Zeilen- oder Seitenumbruchsymbol auf
            einen Takt gezogen. Der Umbruch erfolgt hinter dem
            markierten Takt.

      \subsection{Spationierung}
            Jeder Takt besitzt einen ''Stretch''-Wert der bestimmt,
            wie breit er gesetzt wird. Dieser Wert kann nach dem
            markieren des Taktes durch Eingabe von + vergrößert und
            durch Eingabe von - verkleinert werde. Sind mehrere Takte
            markiert, dann werden alle markierten Takte schmaler oder
            breiter. Mit Hilfe des ''Stretch''-Wertes kann auch der
            Zeilen- oder Seitenumbruch beeinflußt werden.
            Der Stretch-Wert wird durch die Funktion
            Layout->ResetStretch auf den Normalwert zurückgesetzt.


\section{Stimmauszüge}
      Stimmauszüge können erstellt werden, indem nicht gewünschte
      Stimmen (Parts) im Partlisteneditor durch anclicken des Schalters
      ''zeigen'' unsichtbar geschaltet werden.

\section{Styles}

      \Screen{style1}{Style Editor}
      \Screen{textstyle}{Text Style Editor: Styles}
      \Screen{textstyle1}{Text Style Editor: Positions}


\section{\M\ Fileformat}

      \M\ speichert eine Partitur in eine Datei mit dem
      Suffix \type{*.msc}. Dies ist eine normale Textdatei im
      XML Format und kann deshalb mit einem normalen Texteditor
      verändert werden. Zum \M\ Fileformat gibt es keine formale
      Beschreibung (z.B. eine DTD), da es wahrscheinlich
      noch verändert wird, um es den Erfordernissen von neuen
      \M\ Eigenschaften anzupassen. Das Format ist jedoch sehr
      an MusicXML angelehnt. MusicXML sollte verwendet werden, um
      Partituren zu portieren, sollte sich das \M\ Format verändern.

      Das \M\ Dateiformat hat folgende Struktur:

            \starttyping
            <?xml version="1.0"?>
            <museScore version="1.0">
            \stoptyping


\chapter{Installation}

      \section{Download}
            Die neuste \M\ Version gibt es bei
            \type{http://mscore.sourceforge.net}

            \M\ kommt als komprimiertes TAR File und muß
            nach dem Download ausgepackt werden:\par

\starttyping
tar xvofj mscore-0.4.0.tar.bz2
\stoptyping

            Dieses Kommando erzeugt ein Unterverzeichnis
            mscore-0.4.0, in den der \M\ Sourcode kopiert
            wird.

   \section{Voraussetzungen}
      Einige Linux-Distributionen splitten Pakete in einen Anwenderteil und
      einen Entwicklerteil. Zum übersetzen von \M\ benötigen sie
      beide Teile.

            \startitemize[packed]
                  \item cmake $\ge$ 2.4
                  \item qt4 Gui Library in der Version $\ge$ 4.2.x mit
                        freetype2 support (Compiliert mit der Option \type{-xft})
                  \item ALSA Version 0.9.0 oder neuer für MIDI Eingaben
                  \item JACK Audio Server
                  \item fluidsynth-1.0.0 und einen geeigneten SoundFont
                  \item einen aktuellen X11 Server mit freetype2 Support
                        und Render Extension (für Bildschirmfonts).
            \stopitemize

   \section{Compilieren}

   \subsection{Konfigurieren}

   \subsection{make}

\starttyping
make
\stoptyping

      compiliert die Source-Dateien und erzeugt die ausführbare Datei
      \type{mscore}.

      \FrameTitle{Achtung}
      \StartFrame
      Zwischen verschiedenen C++ Compilern gibt es kleine Unterschiede im
      ABI (Application Binary Interface), die zu Inkompatibilitäten führen
      können. Deshalb sollte \M\ mit dem gleichen Compiler compiliert werden,
      mit dem die Qt-Library compiliert wurde.
      \StopFrame

   \section{Installieren}

\starttyping
su -c make install
\stoptyping

      installiert \M.

\chapter{Tastaturabkürzungen}

      \Sc{\Uparrow} Großbuchstabenumschaltung (Shift)
      \blank[line]

      \startcolumns[n=2,rule=on]
      \Sc{Alt+F} File Menü\par
      \Sc{Alt+E} Edit Menü\par
      \Sc{Alt+C} Create Menü\par
      \Sc{Alt+N} Notes Menü\par
      \Sc{Alt+L} Layout Menü\par
      \Sc{Alt+S} Style Menü\par
      \Sc{Alt+D} Display Menü\par
      \Sc{Alt+H} Hilfe Menü\par
      \blank[small]
      \hrule
      \blank[small]
      \Sc{A}     Note a\par
      \Sc{B}     Note b\par
      \Sc{C}     Note c\par
      \Sc{D}     Note d\par
      \Sc{E}     Note e\par
      \Sc{F}     Note f\par
      \Sc{G}     Note g\par
      \blank[small]
      \hrule
      \blank[small]
      \Sc{\Uparrow A} Note a hinzufügen\par
      \Sc{\Uparrow B} Note b hinzufügen\par
      \Sc{\Uparrow C} Note c hinzufügen\par
      \Sc{\Uparrow D} Note d hinzufügen\par
      \Sc{\Uparrow E} Note e hinzufügen\par
      \Sc{\Uparrow F} Note f hinzufügen\par
      \Sc{\Uparrow G} Note g hinzufügen\par
      \blank[small]
      \hrule
      \blank[small]
      \Sc{1}       Prime nach oben\par
      \Sc{2}       Sekunde nach oben\par
      \Sc{3}       Terz nach oben\par
      \Sc{4}       Quarte nach oben\par
      \Sc{5}       Quinte nach oben\par
      \Sc{6}       Septe nach oben\par
      \Sc{7}       Septime nach oben\par
      \Sc{8}       Oktave nach oben\par
      \Sc{9}       None nach oben\par
      \blank[small]
      \hrule
      \blank[small]
      \Sc{\Uparrow 1} Prime nach unten\par
      \Sc{\Uparrow 2} Sekunde nach unten\par
      \Sc{\Uparrow 3} Terz nach unten\par
      \Sc{\Uparrow 4} Quarte nach unten\par
      \Sc{\Uparrow 5} Quinte nach unten\par
      \Sc{\Uparrow 6} Septe nach unten\par
      \Sc{\Uparrow 7} Septime nach unten\par
      \Sc{\Uparrow 8} Oktave nach unten\par
      \Sc{\Uparrow 9} None nach unten\par
      \blank[small]
      \hrule
      \blank[small]
      \Sc{Right}  nächste Note\par
      \Sc{Left}   Note zurück\par
      \Sc{Alt+Up}\crlf selektiert höhere im Akkord oder niedrigste
                  Note der höheren Zeile\par
      \Sc{Alt+Down}\crlf selektiert niedrigere Note im Akkord oder die höhere
                Note der nächsten Zeile.\par
      \Sc{Alt+Strg+Up}  \crlf selektiert die höhere Note im Akkord\par
      \Sc{Alt+Strg+Down} \crlf selektiert die niedrigste Note im Akkord\par
      \stopcolumns

\chapter{Design and Implementation}

   \section{Main Structure}

      \M\ handles multiple documents (scores). The documents can
      selected with a tab bar. A global pointer points to the current
      selected score (\type{cs}).

      A \type{Score} contains data read from an \type{*.msc} or \type{*.xml}
      file. Think of it as a simple list of measures. The \type{layout()}
      procedure breaks this list into lines ans pages producing a
      \type{Layout} structure. The \type{Layout} can be viewed by a
      \type{Viewer} providing a magnification and horizontal und
      vertical offsets.

      \setupFLOWcharts
        [dx=2\bodyfontsize,
         dy=2\bodyfontsize,
         width=7\bodyfontsize,
         height=3\bodyfontsize,
         maxwidth=\textwidth]

      \startFLOWchart[mops1]

        \startFLOWcell
          \name{MuseScore}
          \location{1,1}
          \shape{procedure}
          \text{MuseScore}
          \connect[bl]{Project1}
          \connect[bl]{Project2}
          \stopFLOWcell

        \startFLOWcell
          \name{Project1}
          \location{1,2}
          \shape{32}
          \text{Score\\A}
          \connect[rl]{Layout1}
          \connect[rl]{Layout2}
          \stopFLOWcell

        \startFLOWcell
          \name{Layout1}
          \location{2,2}
          \shape{34}
          \text{Layout}
          \connect[rl]{View1}
          \stopFLOWcell

        \startFLOWcell
          \name{View1}
          \location{3,2}
          \shape{17}
          \text{View PagePreView}
          \stopFLOWcell

        \startFLOWcell
          \name{Layout2}
          \location{2,3}
          \shape{34}
          \text{Layout}
          \connect[rl]{View2}
          \connect[rl]{View3}
          \stopFLOWcell

        \startFLOWcell
          \name{View2}
          \location{3,3}
          \shape{17}
          \text{View\\Canvas}
          \stopFLOWcell

        \startFLOWcell
          \name{View3}
          \location{3,4}
          \shape{17}
          \text{View Navigator}
          \stopFLOWcell

        \startFLOWcell
          \name{Project2}
          \location{1,3}
          \shape{32}
          \text{Score\\B}
          \stopFLOWcell

      \stopFLOWchart
      \FLOWchart[mops1]

      A \type{Score} can have more than one \type{Layout}. This is used in
      the \type{Page Settings} dialog for a page preview.

      A \type{Layout} can have more than one \type{View}. The main view
      is called \type{Canvas} and can be used to view and edit the score.
      A second smaller \type{View} ist used by the \type{Navigator} widget.

   \section{Layout Object Hierarchy}

      \setupFLOWcharts
        [dx=1.5\bodyfontsize,
         dy=1.5\bodyfontsize,
         width=8\bodyfontsize,
         height=4\bodyfontsize,
         maxwidth=\textwidth]

      \startFLOWchart[mops2]

        \startFLOWcell
          \name{Page}
          \location{1,1}
          \shape{procedure}
          \text{Page}
          \connect[rl]{System}
          \stopFLOWcell

        \startFLOWcell
          \name{System}
          \location{2,1}
          \shape{procedure}
          \text{System}
          \connect[rl]{Bar}
          \connect[rl]{SStaff}
          \connect[rl]{SText}
          \connect[rl]{Measure}
          \connect[rl]{Bracket}
          \stopFLOWcell

        \startFLOWcell
          \name{Bar}
          \location{3,1}
          \shape{procedure}
          \text{Bar}
          \stopFLOWcell

        \startFLOWcell
          \name{SStaff}
          \location{3,2}
          \shape{procedure}
          \text{SStaff}
          \stopFLOWcell

        \startFLOWcell
          \name{SText}
          \location{3,3}
          \shape{procedure}
          \text{SText}
          \stopFLOWcell

        \startFLOWcell
          \name{Bracket}
          \location{3,4}
          \shape{procedure}
          \text{Bracket}
          \stopFLOWcell

        \startFLOWcell
          \name{Measure}
          \location{3,5}
          \shape{procedure}
          \text{Measure}
          \connect[rl]{Slur}
          \connect[rl]{SlurSegment}
          \connect[rl]{Segment}
          \stopFLOWcell

        \startFLOWcell
          \name{Slur}
          \location{4,5}
          \shape{procedure}
          \text{Slur}
          \stopFLOWcell

        \startFLOWcell
          \name{SlurSegment}
          \location{4,6}
          \shape{procedure}
          \text{SlurSegment}
          \stopFLOWcell

        \startFLOWcell
          \name{Segment}
          \location{4,7}
          \shape{procedure}
          \text{Segment}
          \connect[rl]{Chord}
          \connect[rl]{Rest}
          \connect[rl]{Lyric}
          \stopFLOWcell

          \startFLOWcell
            \name{Chord}
            \location{5,7}
            \shape{procedure}
            \text{Chord}
            \connect[rl]{Note}
            \connect[rl]{Attribute}
            \stopFLOWcell

            \startFLOWcell
              \name{Note}
              \location{6,7}
              \shape{procedure}
              \text{Note}
              \connect[rl]{Accidental}
              \stopFLOWcell

            \startFLOWcell
              \name{Attribute}
              \location{6,8}
              \shape{procedure}
              \text{Attribute}
              \stopFLOWcell

              \startFLOWcell
                \name{Accidental}
                \location{7,7}
                \shape{procedure}
                \text{Accidental}
                \stopFLOWcell

          \startFLOWcell
            \name{Rest}
            \location{5,8}
            \shape{procedure}
            \text{Rest}
            \stopFLOWcell

          \startFLOWcell
            \name{Lyric}
            \location{5,}
            \shape{procedure}
            \text{Lyric}
            \stopFLOWcell

      \stopFLOWchart
      \FLOWchart[mops2]

\chapter{Beispiele}
      \vbox{\placecontent}
      \setuphead[section][placehead=no]

      \definemakeup[examples][
         doublesided=yes,page=yes,pagestate=start
         ]

      \startexamplesmakeup
      \section{Bilder einer Ausstellung -- Promenade, Modeste Mussorgsky}
      \externalfigure[promenade.pdf][width=1.1\textwidth]
      \stopexamplesmakeup

      \startexamplesmakeup
      \section{Invention No1, J.S.Bach}
      \externalfigure[inv1-1.pdf][width=1.1\textwidth]
      \stopexamplesmakeup

      \startexamplesmakeup
      \externalfigure[inv1-2.pdf][width=1.1\textwidth]
      \stopexamplesmakeup

      \startexamplesmakeup
      \section{Invention No6, J.S.Bach}
      \externalfigure[inv6-1.pdf][width=1.1\textwidth]
      \stopexamplesmakeup

      \startexamplesmakeup
      \externalfigure[inv6-2.pdf][width=1.1\textwidth]
      \stopexamplesmakeup

\completeindex
\stoptext
